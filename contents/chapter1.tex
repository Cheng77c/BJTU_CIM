\chapter{绪论}

\section{研究背景和意义}

进入21世纪以来,全球城市化进程持续加速。据联合国统计,2007年世界城市人口首次超过总人口的50\%,预计到2050年该比例将达到68\%\cite{UNDESA2019}。国家统计局数据显示,中国城市化率从1978年的17.9\%快速提升至2023年的65.2\%\cite{NBS2024},城市人口规模和空间范围持续扩张。城市化进程在促进经济社会发展的同时,也深刻改变了人类聚居环境的空间格局与风险结构。

高密度人口聚集和复杂基础设施系统的相互依存,使城市系统在面临自然灾害和突发事件时展现出显著的系统性脆弱性。洪涝、地震、台风、极端高温等自然灾害频发,对城市安全与可持续发展构成严重威胁。近年来,极端天气事件呈现出强度增强、频率增加的趋势。2021年郑州"7·20"特大暴雨事件造成城区大面积内涝,交通系统瘫痪,人员伤亡重大,充分暴露了超大城市在极端降水事件面前的系统性风险\cite{StateCouncil2022}。类似事件在全球范围内频繁发生,如2011年东日本大地震引发的复合灾害链\cite{CabinetOffice2012}以及2017年美国休斯敦"哈维"飓风造成的城市洪涝灾害\cite{FEMA2018},均表明现代城市复杂系统在面对极端灾害冲击时存在显著的脆弱性。

在传统的城市治理模式下,灾害管理多以单一部门、静态数据和被动响应为主,难以满足超大城市面对极端灾害时的实时感知与动态调度需求。伴随信息技术演进与治理理念更新,从数字城市向智慧城市的转型成为全球城市发展的核心趋势\cite{Batty2018,Cao2023}。智慧城市理念的提出标志着治理逻辑由静态的数据中心化转向动态的系统联动与行为驱动,强调"感知—分析—响应—恢复"的闭环机制。城市韧性治理作为现代城市管理的重要理论范式,强调通过系统性、适应性与协同性机制提升城市应对复杂风险的能力\cite{CN_Li2024CoastalCIM,CN_Qin2024ResilienceIndex}。
 
近年来,中国在国家和地方层面持续推进韧性城市建设。《韧性城市体系构建纲要(2023--2035年)》明确提出要构建以CIM为核心的城市安全运行平台\cite{NDRC2023};《山东省城市应急管理行动计划(2022--2025年)》重点部署滨海城市风暴潮与洪涝灾害的数字化治理体系\cite{Shandong2022};住房和城乡建设部发布的相关指南也要求在应急设施中推广BIM/CIM协同应用\cite{MOHURD2022}。这些政策为本研究提供了重要的制度背景和应用需求,凸显了面向实际治理场景的技术创新迫切性。与此同时,多地积极探索以CIM驱动的综合应急指挥体系和协同治理模式\cite{CN_Tang2024EmergencyPlatform,CN_Lan2024CIMGovernance}。

城市信息建模(City Information Modeling, CIM)作为智慧城市技术体系的核心组成,是指构建覆盖城市空间全域、对象多尺度、语义多层级的综合性信息表达框架。CIM不仅整合了建筑信息模型(BIM)的精细化建筑语义与地理信息系统(GIS)的空间分析能力,还融合了物联网感知数据、社会经济统计信息和实时运行状态,形成城市系统的数字镜像。从技术架构来看,CIM具备四个核心特征:多源异构数据的融合能力、多尺度空间表达的统一性、多层级语义关联的完整性以及多时态动态过程的可模拟性。在应用层面,CIM已广泛应用于城市规划建设、基础设施管理、公共安全应急、环境监测预警等领域,特别是在城市灾害风险管理中展现出独特价值。通过构建高精度的城市三维数字底座,CIM能够支持洪水淹没模拟、建筑脆弱性评估、疏散路径优化、应急资源调度等关键功能的集成运行,为城市韧性治理提供技术支撑\cite{CN_Zhang2023DigitalTwin,CN_Shi2022CommunityTwin}。

随着信息技术的不断演进,海量城市数据的获取与存储已不再是瓶颈,真正的挑战在于如何实现跨部门、跨领域、跨空间尺度的数据融合与建模。传统的地理信息系统(GIS)侧重于空间数据的可视化与空间分析,而建筑信息模型(BIM)则聚焦于建筑单体及设施层面的精细化信息,两者在应用场景、数据结构和服务对象上各自发展,缺乏统一的语义体系和逻辑框架。这种割裂导致城市在灾害风险建模和应急响应时难以实现多尺度联动\cite{CN_Xu2023DataFusion,CN_Ding2023ModelIntegration}。

基于上述CIM的架构特征与应用价值,其在解决数据异构与平台分裂问题方面展现出显著优势。CIM通过构建统一的信息模型和语义框架,为城市治理提供了系统性的解决方案:在技术层面,打破了BIM与GIS之间的数据壁垒,实现了建筑单体到城市空间的无缝连接;在应用层面,将静态的规划设计与动态的运行管理有机结合,为灾害预测、风险评估和应急疏散提供可计算、可模拟的平台;在治理层面,促进了跨部门、跨行业的协同合作,推动城市治理从"部门本位"向"整体治理"转变。

在城市灾害风险评估研究方面,国内外学者已开展了大量工作。在国际上,基于数字技术的城市灾害风险评估研究起步较早,主要围绕模型精度提升、多源数据融合与不确定性量化等方向展开。欧美学者在城市洪水动力学建模领域取得重要进展,英国学者基于浅水方程构建了二维城市洪水模型,美国学者进一步发展了三维计算流体动力学(CFD)模型,通过求解雷诺平均纳维-斯托克斯(RANS)方程,实现对复杂城市环境中垂向流动的精确描述。德国学者提出了基于遥感数据、地面观测与数值模型融合的洪水风险评估框架,通过数据同化技术提高了风险预测的精度\cite{Amirebrahimi2016}。

中国在城市灾害风险评估方面的研究发展迅速,主要集中在洪涝、地震、台风等主要灾种的风险评估与预警系统建设。清华大学、中科院等科研院所基于SWMM与二维水动力模型的耦合,构建了适用于中国城市特征的内涝风险评估模型。北京师范大学在洪水风险评估的社会脆弱性分析方面做出重要贡献,构建了包含人口密度、经济发展水平、基础设施完善程度等指标的综合脆弱性评估体系。近年来,深圳前海、上海临港等地持续推进CIM与应急管理融合示范工程,形成了基于数字孪生的城市水务与交通协同调度体系\cite{Hu2021,Liu2022}。

在滨海城市灾害风险研究方面学者们重点关注风暴潮、海平面上升与城市内涝的复合效应。针对沿海城市的特殊地理位置和复杂地形,研究团队开发了精细化潮汐淹没模型,结合城市建筑布局和排水系统特征,评估不同情景下的灾害风险等级。部分研究还探讨了气候变化背景下极端天气事件频率和强度的变化趋势,以及这对城市防灾减灾规划提出的新的挑战\cite{Brown2007,Fewtrell2008,Gallegos2009}。这些研究为城市韧性评估提供了重要的理论依据和技术支撑,但在多尺度耦合、实时性和智能化方面仍有提升空间。

当前,城市灾害管理面临的核心挑战集中体现在数据异构与平台分裂、建模精度与实时性矛盾、空间语义断裂与响应孤岛等方面。各类数据分散在不同部门与平台,缺乏共享机制;现有灾害风险评估方法多依赖二维模型和静态数据,在描述复杂城市环境的三维流动特征时存在局限;传统疏散规划多割裂室内外空间,难以形成连续的疏散策略\cite{CN_Yu2022KnowledgeGraph,CN_Zhong2024CrossDomain}。基于上述挑战,构建基于CIM的城市灾害风险评估与智能疏散方法体系,实现从"被动响应"向"主动控制"、从"经验决策"向"数据驱动"、从"单点优化"向"系统协同"的转变,具有重要的理论价值与现实意义。


\section{国内外研究现状}

\subsection{CIM 与 BIM/GIS 语义互操作基础}

城市信息建模(City Information Modeling, CIM)旨在弥合建筑信息模型(Building Information Modeling, BIM)与地理信息系统(Geographic Information System, GIS)在尺度、语义与应用场景上的割裂,形成面向城市治理与应急管理的“数据—模型—决策”闭环。这一理念在由建筑域走向城市域的研究演化中得到系统化阐述,并在典型案例中被验证为可行路径,为后续城市级多源时空数据与模型集成奠定了整体架构基础\cite{Xu2014CIM}。在这一框架下,CIM 不仅被视为几何与语义的载体,也被视为承接业务规则与决策逻辑的“中枢”,强调从项目级 BIM 向城市级“数字底座”的纵向拓展。  

在具体的数据与语义模型路径上,研究者多从 IFC 与城市三维模型的耦合入手,提出面向 3D 城市 GIS 的统一建筑模型构想,强调几何与语义的一体化表达\cite{ElMekawy2012UBM}。相关工作在 IFC 几何构件与 CityGML 建筑单元之间建立映射关系,探索在尺度转换过程中保持拓扑一致性与语义完整性的技术路线。随后,一系列工作围绕 IFC 到 CityGML 的双向转换建立了较完整的技术谱系:既包括“近无损”转换及其质量评估框架,也包括可复用的规则库与图转换方法,用以显式描述几何/语义映射规则和异常处理流程。在复杂空间语义方面,以产权与使用功能为代表的三维权属信息被证明可以在 BIM 基础数据模型上进行扩展,支撑三维权属管理、城市更新评估等城市治理场景\cite{Atazadeh2017Ownership3D}。这些进展共同夯实了 CIM 在“构件级语义—城市级对象—业务规则”三层上的互操作基础,使得同一底层模型能够被规划、建设、运维和应急管理等多部门、多业务复用。

\subsection{室内外一体化通行网络与应急疏散建模}

面向室内外一体化通行与路径规划,BIM 导向的多用途几何网络模型(Multi-purpose Geometric Network Model, MGNM)提出了从 IFC 语义自动抽取室内通行网络的技术路线:通过识别房间、门、走廊、楼梯等语义对象及其邻接关系,构建与建筑几何高度一致的网络拓扑;再将其与室外道路网通过“入口—界面—街廊”等关键要素完成连通,从而实现室内外通行空间的无缝衔接\cite{Teo2016BIMIndoorNetwork}。在此基础上,统一的室内—室外网络模型显著降低了传统接口简化带来的可达性偏差,为室内外联合路径规划、可达性分析和拥堵识别提供了统一的网络表达\cite{Claridades2021SeamlessNav}。相关研究表明,当室内复杂几何通过 BIM 自动抽取为拓扑网络后,跨建筑、街区乃至城市尺度的通行分析可以在统一模型上完成,推动跨场域导航由可行性验证走向流程化落地。

针对应急疏散场景,室内外一体化网络不仅用于最短路径规划,还被进一步嵌入到疏散时间评估、安全瓶颈识别和疏散策略优化之中。最新综述强调三维室内环境的几何、拓扑与语义三元并重,提出以 IFC/室内拓扑的自动抽取作为仿真前置步骤,以稳定供给高质量模型输入\cite{Xie2022AutCon3DIndoorEvac}。在设计—仿真—合规一体化方面,宏观疏散模型能够直接消费 IFC 语义生成通行网络,并自动执行关键法规条文的校核,例如出口数量与分布、疏散距离与时间等,从而显著降低建模与校核成本\cite{Zhu2018IFCEvac}。这类工作为“从设计模型直接生成疏散评估报告”的工程化流程提供了可行支撑。

\subsection{群体行为理论与微观疏散模拟方法}

群体行为与应急疏散的理论基底方面,社会力模型(social force model)以简洁的微观相互作用刻画行人之间以及人与环境之间的力学关系,能够自然再现自组织排队、瓶颈拥堵与“越急越慢”等现象,奠定了微观疏散模拟的理论基石\cite{Helbing1995SocialForce}。在此基础上,关于“逃生恐慌”的典型研究揭示了从层流到间歇性“停—走”再到湍动拥堵的相变机制及其与出口几何、密度阈值之间的定量关系,为通行断面设计与组织管理提供了可检验的物理解释\cite{Helbing2000EscapePanic}。后续的实证研究则通过视频观测与实验室控制试验,进一步量化了灾难人群动力学中的临界密度、速度波动和失稳特征,补强了社会力模型的经验基础\cite{Helbing2007CrowdDisasters}。

与连续模型相对,基于元胞自动机(Cellular Automata, CA)的离散方法在网格空间上刻画行人状态转移,以更新规则近似群体相互作用,具备实现简单、计算开销可控的优势,适于大尺度人群通行与疏散过程的高效模拟\cite{Burstedde2001FloorField}。在工程应用中,社会力模型与 CA 模型常被结合或混合使用,以兼顾行为逼真度与计算效率。随着 BIM 语义抽取和网络生成技术的成熟,建筑几何与功能分区可以直接映射为疏散模拟网格与拓扑结构,实现从建筑设计到疏散仿真的全链路自动化\cite{Zhu2018IFCEvac}。这一方向为将“行为科学—物理模型—建筑语义”统一到同一计算框架下提供了技术基础,也为后续嵌入不同行为假设、风险偏好与应急指挥策略留出了接口。

\subsection{城市洪涝二维水动力模型与实时预报体系}

在城市洪涝风险评估方面,二维浅水方程(Shallow Water Equations, SWE)长期作为主力水动力内核,用于刻画地表径流与淹没过程。传统完整动量方程在高分辨率大范围模拟中往往面临计算成本过高的问题,因此“局部惯性”(local inertial)近似成为兼顾效率与精度的关键路径:通过忽略部分加速度项,在显著降低计算量的同时,仍能保持对亚临界水动力过程的合理刻画,使得在城市或区域尺度开展高分辨率淹没模拟成为可能\cite{Bates2010Inertial}。其适用性研究系统比对了在不同糙率、坡降与边界条件下该近似与全动量方程的误差界,给出了在城市、河网和平原等不同地貌下的模型选型建议\cite{deAlmeida2013LocalInertial}。

围绕模型能力与透明度,基准化研究通过标准算例与观测资料对多种二维水动力模型开展结构化对比,并在数值稳定性、误差统计指标、计算效率等方面建立统一的评估框架,从而提升了不同模型间的可比性与选型透明度\cite{Hunter2008Benchmark2D}。另一方面,面向预报业务的研究则在保证关键物理过程刻画能力的前提下,提出结构简洁、计算高效的二维淹没模型与降阶方案,为实时或近实时的城市洪涝预报奠定基础\cite{Dottori2013Simple2D}。这些进展使得在复杂城市地表条件下进行多场景、多情景的快速情景演练成为可能,为风险图编制、排水系统设计与应急预案制定提供了技术支撑。

\subsection{遥感—模型一体化洪涝同化与不确定性刻画}

随着业务部门对实时性与不确定性刻画的需求提升,遥感—模型一体化逐渐成为增强洪涝预报稳定性和可信度的关键路径。合成孔径雷达(Synthetic Aperture Radar, SAR)具备全天候、穿云与夜间观测优势,能够在极端天气条件下提供洪水淹没范围与水线位置等关键信息,用于同化约束水动力模型。一方面,从高分辨率 SAR 影像近实时自动抽取用于同化的水位或淹没边界样本,显著缩短了“观测—处理—同化—预报”的信息链路,为滚动预报与应急推演提供了高频观测输入\cite{Mason2010TerraSARX,Mason2014DoubleScattering};另一方面,系统综述工作从数据质量、分类方法、后验不确定性评估和流程自动化等维度梳理了 SAR 洪涝反演的发展脉络,指出在城市高层建筑、植被覆盖与强散射背景下的特有挑战,并归纳了相应的解决路径\cite{Giustarini2015UncertaintySARFlood,Shen2019RemoteSensingFlood,Li2019RSUrbanSARFusion}。

在“观测—同化—预报”的闭环中,顺序同化 SAR 衍生淹没图被证实可以显著降低模拟水位误差并改善洪泛边界位置,尤其有利于提高对局部积水与堤防漫顶等敏感区域的识别能力\cite{Hostache2018WRR}。进一步地,将概率型淹没图纳入耦合水文—水动力模型,并采用调温粒子滤波等先进同化算法抑制样本退化,可在较长滚动窗口内稳定提升径流与水位联合预报性能,实现从观测概率到状态估计的统一不确定性描述框架\cite{DiMauro2021HESS,DiMauro2022WRR}。这一方向为将“观测误差—模型误差—决策风险”贯通起来提供了方法基础,也为后续多源观测(如水位计、众包观测等)与多模型集合预报的集成留出了空间。

\subsection{城市水系统数字孪生与“模型—数据—治理”闭环}

在“模型—数据—治理”的落地层面,数字孪生(Digital Twin)被视为连接多源观测、物理/数据驱动模型与协同决策的核心承载体。流域尺度的前沿研究提出了面向可迁移、可持续与公平治理的数字孪生框架,明确了数据、模型、政策与能力建设之间的协同要求与挑战清单,并通过防灾减灾、资源调配与长期情景分析等示例展示了其在提升流域韧性和多方协同决策中的应用潜力\cite{Yang2024npjNH}。在工程与运维方向,水务系统数字孪生相关研究总结了其在实时数据采集与融合、状态估计与同化、主动控制与优化以及降阶加速等方面的关键要素与评估指标,强调应通过运营闭环的性能提升和成本收益分析来检验孪生系统的实际价值\cite{Cavalieri2024SensorsAAS,Bonilla2022WaterDTWDS}。

综合来看,从 CIM 底座与语义互操作,到室内外一体化网络与疏散行为模型,再到二维洪涝主干模型与 SAR 同化,以及城市水系统数字孪生的工程化路线,国内外研究已基本形成面向“风险识别—疏散响应—治理优化”闭环的技术栈。然而,要实现跨城市、跨区域的规模化落地,仍需在高保真 BIM/GIS 转换流水线与质量控制、跨尺度与跨模型链的不确定性传递,以及“观测—行为—调度”的统一目标函数与实时控制体系上进一步完善,并探索在现有管理体制与技术条件下的可行实施路径。


\subsection{研究理论基础}

为构建面向城市综合防灾减灾与应急管理的韧性治理技术体系,本文主要立足于三个层面的理论基础:城市韧性治理理论、智慧城市与数字孪生理论以及城市信息建模与 BIM--GIS 融合理论。三者分别从治理范式、数字基础设施与数据模型三个维度,为本文提出的“风险评估—智能疏散—构件损伤—协同治理”一体化框架提供支撑。

\subsubsection{城市韧性治理理论基础}

城市韧性治理(urban resilience governance)源于韧性科学与现代治理理论的综合发展,将城市视为由自然环境、社会经济与技术基础设施共同构成的社会--生态--技术复合系统(Social--Ecological--Technical Systems, SETS)。韧性理论经历了工程韧性、生态韧性与演化韧性等多个发展阶段:工程韧性强调系统在遭受扰动后的快速恢复,生态韧性关注系统在跨稳态条件下维持关键功能的能力,而演化韧性则突出系统在持续变化背景下通过学习、创新与制度重构实现“向前恢复”(bounce forward)的能力。

在此基础上,城市综合韧性可抽象表示为
\begin{equation}
R(t) = f\big(C_{a}(t),\, C_{r}(t),\, C_{t}(t)\,\big|\,G(t)\big),
\end{equation}
其中,$C_{a}(t)$ 为吸收能力,刻画城市在冲击发生时削弱损失的能力;$C_{r}(t)$ 为自适应能力,反映系统在不确定情景下调整结构与功能的能力;$C_{t}(t)$ 为转型能力,表示在深度变革压力下进行路径重塑与结构升级的潜力;$G(t)$ 则代表治理体系的制度安排、主体网络、决策机制与知识系统。该表达式揭示了治理结构在韧性生成中的核心调节作用。

在各类自然灾害和突发事件情景下,城市韧性治理强调从单一部门、单一灾种的“事后响应”模式,转向覆盖“事前预防—事中应对—事后恢复”的全过程、多主体协同模式。一方面,基于脆弱性与暴露度的框架,城市风险水平不仅取决于外部扰动强度,还受暴露对象、敏感性以及适应能力的共同影响;另一方面,通过多源数据与模拟模型支持的情景推演,可以在规划、建设与运营阶段提前识别关键基础设施和人员密集区的薄弱环节,从而提升系统整体的吸收、自适应与转型能力。

本文在城市韧性治理理论框架下:(1)在城市、建筑与构件多尺度上刻画灾害风险演化过程,为识别易损单元提供依据;(2)通过室内外一体化疏散网络与动态路径规划,强化应急响应阶段的人群安全保障能力;(3)结合构件级损伤评估与功能恢复分析,为灾后修复与系统转型提供量化支撑,从治理视角实现对 $C_{a}(t)$、$C_{r}(t)$ 与 $C_{t}(t)$ 的系统性提升。

\subsubsection{智慧城市与数字孪生理论基础}

智慧城市(smart city)与数字孪生(digital twin)为城市韧性治理提供了关键的数字基础与技术路径。智慧城市以新一代信息通信技术(ICT)、物联网(IoT)与大数据分析为核心,通过对城市运行状态的感知、传输、计算与反馈,实现城市基础设施和公共服务的智能化与精细化管理。数字孪生在此基础上进一步发展,将物理城市与虚拟城市通过数据流和模型体系紧密耦合,使城市在虚实互动的闭环结构中开展状态监测、趋势研判、情景模拟与优化决策。

从形式化角度,可将智慧城市能力与数字孪生能力分别抽象为
\begin{equation}
S(t) = g\big(I(t),\, T(t),\, G(t),\, C(t)\big),
\end{equation}
\begin{equation}
D(t) = h\big(M(t),\, U(t),\, B(t)\big),
\end{equation}
其中,$S(t)$ 表示城市在时间 $t$ 的智慧化水平,$I(t)$ 代表感知基础设施,$T(t)$ 表示计算与通信平台能力,$G(t)$ 表征治理结构,$C(t)$ 表示公民参与与社会资本;$D(t)$ 表示数字孪生能力,$M(t)$ 为多模型体系,$U(t)$ 表示数据驱动更新的频率与精度,$B(t)$ 代表虚实双向耦合的有效性。城市整体治理绩效可进一步表示为
\begin{equation}
P(t) = F\big(S(t),\, D(t)\big),
\end{equation}
其中 $P(t)$ 反映城市在安全性、可靠性、运行效率与可持续性等方面的综合表现。该框架表明,智慧城市与数字孪生并非彼此独立的概念,而是共同嵌入城市治理体系的双重能力结构:前者构建数据与平台基础,后者则将数据与模型转化为面向具体场景的预测与决策能力。

在城市公共安全场景下,数字孪生城市强调利用高精度三维模型与多源数据,实现对人群活动、基础设施运行状态及多类风险传播过程的动态仿真,并通过可视化与交互分析支撑应急决策。本文提出的基于 CIM 的多尺度风险评估、室内外一体化疏散与构件级损伤评估,实质上是在智慧城市与数字孪生理论框架下,面向城市综合风险治理构建的一套场景化应用:通过数字孪生场景中对风险演化、人群疏散行为和关键设施损伤的联动推演,为应急指挥部门提供“可感知、可计算、可推演、可追溯”的决策支撑环境。

\subsubsection{城市信息建模与 BIM--GIS 融合理论基础}

城市信息建模(City Information Modeling, CIM)作为智慧城市与数字孪生的重要基础设施,是整合多源空间数据与语义信息的统一表达框架。CIM 以城市整体三维几何模型为载体,融合建筑信息模型(Building Information Modeling, BIM)、地理信息系统(Geographic Information System, GIS)、物联网传感数据及统计业务数据等多源异构信息,构建跨尺度、跨领域的城市数字底座。其典型技术架构包括数据采集层、数据管理层和应用服务层:前者负责获取地形地貌、建筑设施、环境要素与人群活动等数据,中间层面向多源异构数据开展存储、索引与质量控制,应用层则提供三维可视化、空间分析、模拟预测与决策支持等功能。

BIM 侧重于建筑与基础设施层面的精细化信息表达,能够描述构件的几何形态、材料特性、构造关系与生命周期属性;GIS 则擅长城市与区域尺度的空间分析与场景表达,强调地形地貌、土地利用、交通网络与环境要素等多维空间关系。BIM 与 GIS 在空间精度、语义层级和应用尺度上具有显著互补性,通过二者的深度融合,可在建筑构件—建筑单体—街区—城市等多尺度上建立连续、一致的空间信息链条。

在城市综合防灾减灾与应急管理研究中,CIM 以及 BIM--GIS 融合理论主要体现在以下三个方面:(1)在数据层面,打通建筑内部构件信息与城市外部地形、基础设施和道路网络数据,实现从城市空间格局到建筑内部空间的全过程统一建模;(2)在模型层面,为三维模拟分析与室内外疏散网络构建提供高精度几何与语义底图,使风险评估结果能够精确映射到建筑与构件层级,并进一步驱动基于语义的疏散路径规划与损伤评估;(3)在应用层面,通过 CIM 平台实现风险评估、路径诱导与损伤分析等多模型的可视化集成和协同调用,为多部门、多层级、多角色参与的综合应急指挥与韧性治理提供统一支撑环境。

综上,城市韧性治理理论为本文提供了目标与评价框架,智慧城市与数字孪生理论为本文构建多模型耦合与虚实互动的技术路径,城市信息建模与 BIM--GIS 融合理论则为本文实现多源数据集成、三维仿真与一体化决策支持奠定了数据与模型基础。三者共同构成本文研究的理论支撑体系。



\section{研究目标和内容}

本研究旨在构建一个面向滨海城市洪涝场景的CIM驱动灾害风险评估、智能疏散与灾后损伤评估协同框架,形成“数据汇聚—风险计算—路径诱导—损伤复盘—协同指挥”的闭环流程,并以威海滨海应急服务中心为验证场景。围绕这一总体目标,研究重点回答三个相互递进的问题:(1) 如何实现多源数据的语义一致性表达,整合BIM、GIS、遥感与物联网数据并准确刻画风暴潮洪涝的三维动力学演化;(2) 如何构建风险驱动的室内外一体化疏散模型,将动态洪水风险映射到疏散网络,实现语义一致、拓扑连通且可实时更新的路径规划;(3) 如何在统一的CIM平台上实现风险评估、路径诱导与构件级损伤评估的联动,并通过典型滨海案例验证系统的实用性与可靠性。

针对上述问题,本研究设计了四个层层递进的研究内容:一是建立多源数据底座与高精度洪涝风险评估方法,构建三维水动力模拟与脆弱性指标体系,输出时空连续的危险度产品;二是提出MGNM增强策略与风险驱动代价函数,构建室内外联通的疏散网络并实现动态重规划;三是面向灾后恢复建立构件级洪灾损伤评估与三维可视化方法,实现与风险、疏散模型的多尺度联动;四是构建威海滨海应急服务中心综合验证流程,围绕洪涝模拟、疏散规划与构件损伤联动评估体系适用性。

\subsection{研究路线与数据支撑}

本研究遵循"理论分析—方法设计—系统开发—实验验证"的路线展开。数据层面,采集无人机倾斜摄影、激光雷达扫描、楼宇BIM、城市管网GIS、潮位与雨量监测等资料;对象层面,以滨海应急服务中心主体建筑及周边关键设施为研究单元;验证层面,通过历史风暴潮过程与应急演练脚本开展对比验证。每一阶段的核心成果在下一阶段得到继承,实现从机理模型到业务系统的逐步落地。

本研究面向洪水等致灾情景下的人员安全疏散,提出一种以 CIM(City Information Modeling)为核心的室内外一体化疏散与闭环引导系统架构(见图~\ref{fig:tech_architecture})。系统遵循"数据—模型—决策—服务"的流水线思想,自下而上划分为七个层级,并通过标准化数据与控制接口实现解耦与可替换。

\begin{figure}[htbp]
\centering
\includegraphics[width=0.8\textwidth]{PIC/技术架构图.png}
\caption{本文整体研究框架}
\label{fig:tech_architecture}
\end{figure}

\noindent\textbf{L1 数据采集层}:汇聚多源异构数据以保障真实性与时效性,包括:航空/近景数据(无人机倾斜摄影、航测),地形与几何(LiDAR 点云/网格),建筑语义信息(BIM/IFC),城市空间与道路网络(GIS/CityGML/OSM),以及时序监测(潮位、雨量、泵站与闸门状态、人群密度等)。

\noindent\textbf{L2 语义融合与治理(CIM)}:针对入湖前数据执行坐标与高度基准统一、分辨率重采样与时空索引构建,形成城市统一语义模型(建筑—楼层—房间—出入口—道路—广场等实体及拓扑关系),并沉淀于数据湖/仓库,支撑在线与离线计算。

\noindent\textbf{L3 风险评估层}:基于 RANS 的三维水动力模型模拟外部致灾因子演化,输出水深/流速/淹没范围等栅格要素记为 $H(x)$。结合物理、社会、经济维度构建脆弱性指标 $V(x)$,定义复合风险
\begin{equation}
R(x)=f\!\left(H(x),\,V(x)\right),
\end{equation}
常见实现为加权归一化或层次化综合,并可输出风险栅格与等值线以供下游使用。

\noindent\textbf{L4 室内外一体化网络建模}:由 IFC 自动生成多粒度导航网络(MGNM)。通过骨架提取与走廊中心线化(如改进 MAT)得到室内连通图,识别房间/走廊/门窗/楼梯(电梯)等关键节点;再与室外道路/广场网络在出入口—楼梯—街道处拓扑拼接,构建室内外统一可达图。

\noindent\textbf{L5 路径规划与动态引导}:在统一可达图上进行风险感知最短路搜索。对边 $e$ 定义综合成本
\begin{equation}
C(e)=\ell(e)+\lambda\,k_t(e)+\beta\,\rho(e),
\end{equation}
其中 $\ell(e)$ 为几何或时间长度,$k_t(e)$ 来源于人群密度热图与历史吞吐的拥堵惩罚,$\rho(e)$ 为由 $R(x)$ 沿入口/楼梯等结构要素投影或积分得到的风险暴露度,$\lambda,\beta$ 为权重。为兼顾城市级规模与实时性,采用"粗到细/单尺度"可切换策略,并在应急模式下周期刷新 $\tilde{H}(x)$ 与 $k_t(e)$ 实施局部重规划。

\noindent\textbf{L6 应用与可视化层}:面向多角色提供双端服务——指挥与应急响应控制台用于态势总览、告警联动、策略下发与资源调度;公众终端(App/小程序)提供个体/小群体引导与动态提醒;WebGIS/三维可视化展示风险演化、可达性变化与人群分布。

\noindent\textbf{L7 案例与评估}:在真实场景中开展综合演练与对比评估,指标涵盖:总撤离时间、处于高风险边上的路径占比、服务覆盖率、网络冗余度、重规划次数与端到端引导时延等,并结合专家评分与问卷对有效性、可拓展性与适配性进行定量与定性评价。

\noindent\textbf{数据与控制流说明}:系统形成从"采集—治理—评估—建模—规划—引导"的单向数据流与"监测/人流反馈—状态更新—策略下发"的反向闭环控制流。具体为:L1 多源数据经 L2 清洗统一后入湖;L3 结合静态地形与实时水文产生 $H(x)$ 与 $R(x)$;L4 由 BIM/IFC 构建室内网络并与室外拓扑拼接;L5 将 $R(x)$ 投影到网络边并按 $C(e)$ 进行多策略规划,生成指挥端与公众端指令;L6 输出可视化与引导服务;公众与现场终端的位置、速度、拥堵等实时反馈经 L2 更新模型状态,实现闭环优化。权重参数 $\lambda,\beta$ 与阈值的选取依据及灵敏度分析详见第5章。

本研究的主要创新点体现在理论创新、方法创新、技术创新、应用创新四个方面。在理论创新方面,提出基于CIM的城市灾害全过程管理理论框架,建立多源数据融合的统一语义模型,发展覆盖城市、建筑与构件多尺度的空间建模理论;在方法创新方面,建立高精度三维水动力风险评估方法,设计语义驱动的室内外网络自动构建与风险感知疏散算法,并提出结合水动力作用与材料耐水性的构件级洪灾损伤评估方法;在技术创新方面,构建支持实时响应与灾后复盘的集成系统架构,开发面向多用户的可视化交互平台,实现风险评估、路径诱导与损伤评估的一体化应用;在应用创新方面,建立完整的案例验证框架,形成可复制推广的技术方案,推动城市韧性治理能力的系统提升。

本研究对"城市韧性治理"理论的贡献体现在理论层面、方法层面、实践层面三个方面。在理论层面,本研究首次系统性地将CIM技术范式与城市韧性治理理论相结合,提出"数字韧性治理"概念框架;基于复杂系统理论与CIM的多尺度建模能力,发展了多尺度韧性协同治理理论;将动态韧性边界控制理论引入城市治理领域,提出动态韧性边界治理理论。在方法层面,集成数字航空摄影测量、BIM–GIS融合与三维RANS水动力建模,构建具备构件级精度的洪涝风险评估体系;基于MGNM的自动构建与语义增强路径规划算法,实现从IFC到疏散网络的无缝转换;基于CIM平台的统一语义框架,构建了支持多部门、多层级、多角色协同参与的韧性治理方法。在实践层面,通过威海沿海城市典型案例开展全流程验证,构建了从数据采集、模型构建、风险评估到应急响应的完整技术体系;通过将建筑信息学、地理信息科学、水动力学、运筹优化等多学科知识进行系统集成,为跨学科融合研究提供了成功示范;为智慧城市建设提供了韧性导向的发展路径。

综合上述贡献,本研究在理论创新、方法突破与实践示范等方面为城市韧性治理学科发展做出了系统性贡献,对推动韧性治理理论的数字化转型与实践化应用具有重要的学术价值与现实意义。
