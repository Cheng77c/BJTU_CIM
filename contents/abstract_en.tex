% ********************************************************************
% ****************** Free to modify the content below ******************
% ********************************************************************

With accelerating urbanization and frequent extreme weather events, cities face dual challenges of disaster risk assessment and personnel evacuation caused by natural meteorological factors. Traditional risk assessment methods are predominantly based on two-dimensional planar models and lack effective coupling with evacuation planning, failing to meet the refined emergency management needs of modern cities. This research constructs an integrated multi-scale disaster risk assessment and intelligent evacuation framework based on City Information Modeling (CIM), aiming to address key scientific problems in urban emergency management including insufficient risk assessment accuracy, fragmented indoor-outdoor evacuation paths, and poor spatiotemporal model coordination.

First, this study establishes a centimeter-scale urban terrain semantic model by integrating multi-source data from UAV oblique photogrammetry and LiDAR. A nested hydrodynamic solver is developed based on three-dimensional Reynolds-Averaged Navier-Stokes (RANS) equations, utilizing 65.8 million structured grid cells to achieve high-precision flood simulation, thereby providing a reliable physical foundation for urban-scale risk assessment. Second, this study innovatively proposes a Multi-purpose Geometric Network Model (MGNM) that automatically integrates indoor IFC semantics with outdoor road networks within the CIM framework. A unified evacuation network representation is constructed, incorporating risk-aware cost functions that couple dynamic flood parameters with crowd density. Coarse-to-fine parallel path-solving algorithms are developed, optimizing evacuation computation time for five interconnected buildings from 485.2 seconds to 127.6 seconds, thereby improving computational efficiency by 74\% and achieving integrated indoor-outdoor intelligent evacuation path planning. Third, based on Assembly-Based Vulnerability theory, the research extends BIM component semantics, establishing quantitative criteria covering multiple action mechanisms including hydrostatic pressure, hydrodynamic pressure, and buoyancy forces, enabling cross-scale damage assessment from urban to component levels with three-dimensional visualization. Validation results from the Weihai Coastal Emergency Service Center demonstrate overall component damage rates of 27\% (door-window damage rate 62\%, wall surface damage rate 54\%), estimated direct economic losses of CNY 514,000, functional recovery time of 21 days, with the integrated system completing comprehensive "risk assessment—evacuation planning—damage analysis" chain calculations in 14.6 minutes, meeting real-time requirements for emergency command.

This research theoretically enriches the CIM-based urban resilience theory system, proposes collaborative modeling methods for multi-scale risk assessment and intelligent evacuation, and practically provides urban emergency management departments with integrated digital tools encompassing risk warning, evacuation guidance, and loss assessment, demonstrating significant engineering application value and promotion prospects.

