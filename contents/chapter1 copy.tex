\chapter{绪论}

\section{研究背景和意义}

进入21世纪以来,全球城市化进程持续加速。据联合国统计,2007年世界城市人口首次超过总人口的50\%,预计到2050年该比例将达到68\%\cite{UNDESA2019}。国家统计局数据显示,中国城市化率从1978年的17.9\%快速提升至2023年的65.2\%\cite{NBS2024},城市人口规模和空间范围持续扩张。城市化进程在促进经济社会发展的同时,也深刻改变了人类聚居环境的空间格局与风险结构。

高密度人口聚集和复杂基础设施系统的相互依存,使城市系统在面临自然灾害和突发事件时展现出显著的系统性脆弱性。洪涝、地震、台风、极端高温等自然灾害频发,对城市安全与可持续发展构成严重威胁。近年来极端天气事件呈现出强度增强、频率增加的趋势。2021年郑州7·20特大暴雨事件造成城区大面积内涝,交通系统瘫痪,人员伤亡重大,充分暴露了超大城市在极端降水事件面前的系统性风险\cite{StateCouncil2022}。类似事件在全球范围内频繁发生,如2011年东日本大地震引发的复合灾害链\cite{CabinetOffice2012}、2017年美国休斯敦"哈维"飓风造成的城市洪涝灾害\cite{FEMA2018},均表明现代城市复杂系统在面对极端灾害冲击时存在显著的脆弱性。

在传统的城市治理模式下,灾害管理多以单一部门、静态数据和被动响应为主,难以满足超大城市面对极端灾害时的实时感知与动态调度需求。伴随信息技术演进与治理理念更新,从数字城市向智慧城市的转型成为全球城市发展的核心趋势\cite{Batty2018,Cao2023}。智慧城市理念的提出标志着治理逻辑由静态的数据中心化转向动态的系统联动与行为驱动,强调"感知—分析—响应—恢复"的闭环机制。城市韧性治理作为现代城市管理的重要理论范式,强调通过系统性、适应性与协同性机制提升城市应对复杂风险的能力\cite{CN_Li2024CoastalCIM,CN_Qin2024ResilienceIndex}。
 
近年来,中国在国家和地方层面持续推进韧性城市建设。《韧性城市体系构建纲要(2023--2035年)》明确提出要构建以CIM为核心的城市安全运行平台\cite{NDRC2023};《山东省城市应急管理行动计划(2022--2025年)》重点部署滨海城市风暴潮与洪涝灾害的数字化治理体系\cite{Shandong2022};住房和城乡建设部发布的相关指南也要求在应急设施中推广BIM/CIM协同应用\cite{MOHURD2022}。这些政策为本研究提供了重要的制度背景和应用需求,也凸显了面向实际治理场景的技术创新迫切性,同时多地积极探索以CIM驱动的综合应急指挥体系和协同治理模式\cite{CN_Tang2024EmergencyPlatform,CN_Lan2024CIMGovernance}。

城市信息建模(City Information Modeling, CIM)作为智慧城市技术体系的核心组成,是指构建覆盖城市空间全域、对象多尺度、语义多层级的综合性信息表达框架。CIM不仅整合了建筑信息模型(BIM)的精细化建筑语义与地理信息系统(GIS)的空间分析能力,还融合了物联网感知数据、社会经济统计信息和实时运行状态,形成城市系统的数字镜像。从技术架构来看,CIM具备四个核心特征:多源异构数据的融合能力、多尺度空间表达的统一性、多层级语义关联的完整性以及多时态动态过程的可模拟性。在应用层面,CIM已广泛应用于城市规划建设、基础设施管理、公共安全应急、环境监测预警等领域,特别是在城市灾害风险管理中展现出独特价值。通过构建高精度的城市三维数字底座,CIM能够支持洪水淹没模拟、建筑脆弱性评估、疏散路径优化、应急资源调度等关键功能的集成运行,为城市韧性治理提供技术支撑\cite{CN_Zhang2023DigitalTwin,CN_Shi2022CommunityTwin}。

随着信息技术的不断演进,海量城市数据的获取与存储已不再是瓶颈,真正的挑战在于如何实现跨部门、跨领域、跨空间尺度的数据融合与建模。传统的地理信息系统(GIS)侧重于空间数据的可视化与空间分析,而建筑信息模型(BIM)则聚焦于建筑单体及设施层面的精细化信息,两者在应用场景、数据结构和服务对象上各自发展,缺乏统一的语义体系和逻辑框架。这种割裂导致城市在灾害风险建模和应急响应时难以实现多尺度联动\cite{CN_Xu2023DataFusion,CN_Ding2023ModelIntegration}。

基于上述CIM的架构特征与应用价值,其在解决数据异构与平台分裂问题方面展现出显著优势。CIM通过构建统一的信息模型和语义框架,为城市治理提供了系统性的解决方案:在技术层面,打破了BIM与GIS之间的数据壁垒,实现了建筑单体到城市空间的无缝连接;在应用层面,将静态的规划设计与动态的运行管理有机结合,为灾害预测、风险评估和应急疏散提供可计算、可模拟的平台;在治理层面,促进了跨部门、跨行业的协同合作,推动城市治理从"部门本位"向"整体治理"转变。

在城市灾害风险评估研究方面,国内外学者已开展了大量工作。在国际上,基于数字技术的城市灾害风险评估研究起步较早,主要围绕模型精度提升、多源数据融合与不确定性量化等方向展开。欧美学者在城市洪水动力学建模领域取得重要进展,英国学者基于浅水方程构建了二维城市洪水模型,美国学者进一步发展了三维计算流体动力学(CFD)模型,通过求解雷诺平均纳维-斯托克斯(RANS)方程,实现对复杂城市环境中垂向流动的精确描述。德国学者提出了基于遥感数据、地面观测与数值模型融合的洪水风险评估框架,通过数据同化技术提高了风险预测的精度\cite{Amirebrahimi2016}。

中国在城市灾害风险评估方面的研究发展迅速,主要集中在洪涝、地震、台风等主要灾种的风险评估与预警系统建设。清华大学、中科院等科研院所基于SWMM与二维水动力模型的耦合,构建了适用于中国城市特征的内涝风险评估模型。北京师范大学在洪水风险评估的社会脆弱性分析方面做出重要贡献,构建了包含人口密度、经济发展水平、基础设施完善程度等指标的综合脆弱性评估体系。近年来,深圳前海、上海临港等地持续推进CIM与应急管理融合示范工程,形成了基于数字孪生的城市水务与交通协同调度体系\cite{Hu2021,Liu2022}。

在滨海城市灾害风险研究方面学者们重点关注风暴潮、海平面上升与城市内涝的复合效应。针对沿海城市的特殊地理位置和复杂地形,研究团队开发了精细化潮汐淹没模型,结合城市建筑布局和排水系统特征,评估不同情景下的灾害风险等级。部分研究还探讨了气候变化背景下极端天气事件频率和强度的变化趋势,以及这对城市防灾减灾规划提出的新的挑战\cite{Brown2007,Fewtrell2008,Gallegos2009}。这些研究为城市韧性评估提供了重要的理论依据和技术支撑,但在多尺度耦合、实时性和智能化方面仍有提升空间。

当前,城市灾害管理面临的核心挑战集中体现在数据异构与平台分裂、建模精度与实时性矛盾、空间语义断裂与响应孤岛等方面。各类数据分散在不同部门与平台,缺乏共享机制;现有灾害风险评估方法多依赖二维模型和静态数据,在描述复杂城市环境的三维流动特征时存在局限;传统疏散规划多割裂室内外空间,难以形成连续的疏散策略\cite{CN_Yu2022KnowledgeGraph,CN_Zhong2024CrossDomain}。基于上述挑战,构建基于CIM的城市灾害风险评估与智能疏散方法体系,实现从"被动响应"向"主动控制"、从"经验决策"向"数据驱动"、从"单点优化"向"系统协同"的转变,具有重要的理论价值与现实意义。

\section{国内外研究现状}

\subsection{CIM技术理论基础}

城市信息建模(City Information Modeling, CIM)作为智慧城市技术体系的重要组成,旨在构建覆盖城市空间全域、对象多尺度、语义多层级的综合性信息表达框架。CIM不仅是建筑信息建模(Building Information Modeling, BIM)与地理信息系统(Geographic Information System, GIS)融合发展的自然延伸,亦是城市治理由数据静态存储向动态智能响应转型的关键基础。在构成层面,CIM系统主要由五类关键要素构成:几何精度、语义结构、时空动态、数据互通与智能联动。在技术演化路径上,CIM的发展大致经历三个阶段:初期以建筑导向为中心;中期实现由建筑向城市尺度的技术拓展,BIM与GIS的耦合成为核心议题\cite{Cao2023};当前阶段则迈向语义统一、行为建模与平台联通\cite{Fang2024,CSKC202404007,GJDA202404002}。

在CIM构建过程中,高精度三维几何数据获取是基础支撑。传统数字高程模型(DEM)多源自卫星或低分辨率航空影像,空间分辨率常在30m量级,难以满足城市内部复杂地形与建筑结构的精细表达需求。数字航空摄影测量通过对重叠影像进行密集匹配重建三维点云,具有高精度覆盖、自动化程度高、多尺度适应性、成本效益优等优势。搭载高分辨率相机的无人机可获得厘米级地面分辨率,借助Structure from Motion(SfM)与Multi-View Stereo(MVS)算法,可自动完成由影像到三维模型的重建流程。

CIM的架构体系体现了多维度、多层级的技术特征。在数据维度,CIM整合几何数据、语义数据、时空数据和行为数据四大类核心数据,通过统一的数据模型实现跨域融合。在空间维度,CIM涵盖从城市区域到建筑构件的多尺度表达,支持宏观、中观、微观三个层次的空间分析。在时间维度,CIM不仅包含静态的地理环境信息,还集成动态的实时感知数据和历史变化数据,形成时空连续的数字镜像。在应用维度,CIM提供规划设计、建设管理、运行维护、应急响应等全生命周期的技术支撑。

城市系统作为高度复杂的空间实体,具备多主体、多层次、多反馈的典型特征。复杂系统理论(Complex Systems Theory)为理解和建模城市韧性提供了理论框架,强调系统内部要素的非线性相互作用、涌现(emergence)特性与自组织(self-organization)能力。动态韧性边界控制理论(Dynamic Resilience Boundary Control Theory)强调通过实时感知、动态评估与自适应调节维持系统在"安全边界"内运行。城市系统的层次化结构决定了韧性边界控制需要考虑跨尺度协调,包括构件层边界控制、网络层边界控制、系统层边界控制。

CIM作为数字化时代城市建模的重要技术范式,与灾害韧性理论的交叉融合催生了新的理论视角与实践路径。该交叉领域的核心价值在于将抽象的韧性概念通过具象的三维空间模型与语义信息进行操作化表达,实现韧性理论从定性分析向定量建模、从静态评估向动态仿真的转变。CIM技术为韧性理论的操作化提供了三个关键维度的增强:空间韧性的精确化表达、时态韧性的动态化建模、系统韧性的协同化分析。交叉领域催生的"数字韧性"(Digital Resilience)概念,强调通过数字化手段增强城市系统的适应性与恢复力,成为智慧城市与韧性城市建设的重要理论支撑。

\subsection{城市灾害风险评估研究}

城市灾害风险评估研究主要沿着理论方法创新、技术手段提升和应用实践拓展三个维度展开。在理论方法层面,研究者提出了基于深度学习的不确定性量化框架,通过贝叶斯方法整合多源数据,提高风险评估的可信度\cite{Teng2017}。发展了基于概率的风险评估方法,考虑边界条件不确定性的影响,为防洪决策提供了更可靠的科学依据\cite{Domeneghetti2015}。开发了高分辨率全球洪水灾害模型,实现了从城市到区域的多尺度风险评估\cite{Sampson2015}。同时,自动化制图技术的进步使得传统需要大量人工的洪水风险制图过程变得更加高效\cite{Wing2017}。

在技术手段方面,基于物理过程的精细化建模成为主流趋势。在城市洪水动力学建模领域,基于浅水方程的二维城市洪水模型得到广泛应用,进一步发展的三维计算流体动力学(CFD)模型通过求解雷诺平均纳维-斯托克斯(RANS)方程,实现对复杂城市环境中垂向流动的精确描述。提出了基于遥感数据、地面观测与数值模型融合的洪水风险评估框架,通过数据同化技术提高了风险预测的精度。近年来,机器学习技术与传统物理模型的结合成为新的研究热点,通过神经网络等方法加速计算过程,提高模型的实时性。

中国在城市灾害风险评估方面的研究起步相对较晚,但在国家减灾战略推动下发展迅速,形成了具有中国特色的研究路径。在宏观层面,国家层面建立了全国性的灾害风险评估体系,发布了《中国自然灾害风险图集》,为区域防灾减灾规划提供科学支撑。在技术方法层面,清华大学、中科院等科研院所基于SWMM与二维水动力模型的耦合,构建了适用于中国城市特征的内涝风险评估模型。北京师范大学在洪水风险评估的社会脆弱性分析方面做出重要贡献,构建了包含人口密度、经济发展水平、基础设施完善程度等指标的综合脆弱性评估体系。武汉大学、河海大学等机构在城市排水系统优化和海绵城市绩效评估方面开展了深入研究,为城市内涝治理提供了技术支撑。

在应用实践方面,中国各大城市积极探索基于数字技术的灾害风险评估模式。深圳市建立了基于BIM+GIS的内涝风险评估平台,实现了从预警到响应的全链条管理。上海市构建了超大城市暴雨内涝风险评估模型,为城市更新和基础设施建设提供决策支持。广州市开展了气候变化背景下的城市洪水风险评估研究,评估未来不同气候情景下的灾害风险变化趋势\cite{Zhao2022}。近年来,深圳前海、上海临港等地持续推进CIM与应急管理融合示范工程,形成了基于数字孪生的城市水务与交通协同调度体系\cite{Hu2021,Liu2022}。这些实践探索为中国城市灾害风险评估积累了宝贵经验,并为其他国家提供了可借鉴的中国方案。

\begin{table}[htbp]
  \centering
  \caption{国内CIM+应急管理典型案例概览}
  \label{tab:domestic_cases}
  \begin{tabular}{p{3cm}p{5.5cm}p{5.5cm}}
    \toprule
    地区/项目 & 应用场景 & 关键特征与启示 \\
    \midrule
    深圳前海智慧应急平台 & 洪涝预警—BIM/CIM联动排水调度 & 建立室内外统一网格,实时接入雨量和泵站数据,形成洪水三维可视化与多部门联动处置流程\cite{Hu2021} \\
    上海临港数字孪生底座 & 沿海水务调度—风暴潮防御演练 & 构建数字孪生水务平台,实现风暴潮过程模拟与应急演练,支撑临港新片区的城市应急指挥\cite{Liu2022} \\
    威海滨海应急服务中心(本研究) & 风暴潮风险评估—室内外疏散联动 & 在统一CIM框架下耦合三维水动力模拟与MGNM疏散网络,实现“风险预测—路径诱导—协同指挥”的一体化闭环 \\
    \bottomrule
  \end{tabular}
\end{table}

当前城市灾害风险评估研究呈现出几个重要发展趋势:一是从单一灾种向多灾种耦合转变,关注灾害链效应和复合灾害风险\cite{Kreibich2017};二是从静态评估向动态仿真转变,强调灾害过程的时间演化特征;三是从经验判断向数据驱动转变,利用大数据和人工智能技术提升评估精度;四是从部门分割向协同治理转变,构建跨部门、跨区域的风险管理机制\cite{Liu2020,Wang2021}。这些发展趋势反映了城市灾害风险评估从传统的技术手段向更加综合、智能和协同的方向演进。

尽管国内外研究取得重要进展,但仍存在模型精度与计算效率的矛盾、多源数据融合标准不统一、风险评估与应急响应脱节等共性问题。三维精细化模型虽然能够提供更准确的物理描述,但计算复杂度高,难以满足实时应用的需求;不同数据源在坐标系统、精度等级、更新频率等方面存在差异;现有风险评估多停留在科学分析层面,与实际的应急管理业务流程结合不够紧密。

\subsection{应急疏散与智慧治理研究}

应急疏散与智慧治理作为城市韧性管理的重要组成部分,近年来在理论方法、技术手段和应用实践等方面都取得了显著进展。

在理论研究方面,学者们从传统的静态疏散规划向动态适应性疏散理论转变。基于行为动力学的疏散行为模型考虑了个体决策过程中的认知偏差、社会影响和环境信息等因素\cite{Lovreglio2016}。人群行为建模从理想化的理性人假设向更加真实的行为描述转变,融入了恐慌、从众、群体凝聚力等心理因素。疏散动力学理论通过分析人群流动的基本规律,建立了描述疏散过程速度、密度和流量关系的数学模型。开发了考虑人群异质性和环境动态变化的智能疏散理论框架,为智慧疏散提供了理论基础\cite{Lin2017}。在多目标优化理论方面,研究者将疏散时间、拥堵程度、安全风险等多个目标进行综合考量,构建了更加全面的疏散策略评估体系。

在技术发展方面,人工智能、大数据、物联网等新兴技术与传统疏散理论的融合为智慧疏散提供了新的技术支撑。智能疏散引导系统整合了多源数据、实时通信和路径优化算法,能够为疏散者提供个性化的疏散建议\cite{Li2019}。计算机视觉技术通过监控摄像头实时分析人群密度和流动方向,为疏散决策提供现场数据支撑。无线传感器网络技术可以获取建筑物内部的环境参数,为风险评估提供实时信息。移动通信和定位技术使得大规模人群的位置跟踪和行为分析成为可能。机器学习算法通过分析历史疏散数据,能够预测不同情景下的疏散模式和瓶颈点。强化学习技术在动态路径优化中的应用实现了对环境变化的实时响应和自适应调整。

在应用实践方面,基于BIM和数字孪生技术的疏散仿真系统得到广泛应用,为建筑和城市应急管理提供了决策支持工具\cite{Shi2020}。大型公共场所如体育场馆、机场、地铁站等建立了完善的疏散预警和引导系统,通过声光指示、移动应用推送等方式提供疏散指导。智慧校园和智慧社区项目将疏散管理与日常安防系统相结合,实现了常态与应急状态的平滑切换。基于游戏化的疏散训练系统提高了公众的应急意识和自救能力。城市级的应急指挥平台集成了疏散模拟、资源调度和通信协调等功能,形成了完整的应急管理体系。跨部门协同演练平台使得消防、公安、医疗等部门的联合疏散行动更加高效有序。

在标准规范方面,各国和地区不断完善建筑设计和消防规范中的疏散要求。疏散通道宽度、出口数量、楼梯容量等技术标准得到细化和完善。性能化防火设计方法允许通过计算机仿真验证建筑疏散设计的合理性,为创新性建筑方案提供了技术支撑。绿色建筑评价体系开始纳入应急疏散性能指标,推动了建筑安全与可持续性的协调发展。国际标准化组织发布了应急管理和业务连续性的相关标准,为组织机构的疏散准备提供了指导框架。

这些进展反映了应急疏散研究从经验驱动向数据驱动、从静态规划向动态适应、从单一技术向系统集成、从被动响应向主动预防的发展趋势。未来的研究重点将更加注重多模态数据融合、人工智能辅助决策、虚拟现实培训技术和跨学科协同创新等方面的发展。

城市应急疏散路径规划是运筹学、计算机科学与城市规划学交叉的重要研究领域。经典算法包括Dijkstra最短路径算法\cite{Dijkstra1959}、A*启发式搜索算法\cite{Hart1968}等,这些算法在静态网络环境下能够快速找到最优路径,但在动态变化的灾害环境中适应性有限。为适应动态环境,遗传算法、蚁群优化、粒子群优化等元启发式算法被广泛应用于多约束条件下的路径优化\cite{Dorigo1997}。近年来,强化学习在动态路径规划中的应用受到关注\cite{Gao2021}。

人群疏散行为建模是应急疏散研究的核心内容,主要包括宏观流动模型与微观个体模型两类。社会力模型(Social Force Model)是最具影响力的微观模型之一\cite{Helbing1995}。元胞自动机(Cellular Automata)模型通过离散的空间-时间-状态描述个体运动\cite{Blue1998}。多智能体系统(Multi-Agent System)为个体行为建模提供了更加灵活的框架。

传统疏散研究多分别针对室内环境或城市道路网络,缺乏统一的建模框架。随着BIM与GIS技术的发展,室内外一体化疏散建模成为新的研究热点。通过从IFC模型中提取空间拓扑信息,构建室内导航网络\cite{Teo2016}。多用途几何网络模型(MGNM)为室内复杂空间的网络建模提供了有效方法。

传统"数字城市"强调静态信息集成与平台可视化,而"智慧城市"理念标志着治理逻辑由静态的数据中心化转向动态的系统联动与行为驱动。在此框架下,城市被视为多主体、多系统、多反馈的复杂适应系统,治理强调"感知—分析—响应—恢复"的闭环机制。CIM被赋予新的关键角色,具备多尺度空间表达、多语义关系建模与动态状态感知能力,可支撑灾害全过程响应。

当前应急疏散研究仍面临空间语义断裂、动态适应性不足、与风险评估脱节等挑战。室内BIM模型与室外GIS数据在语义表达、坐标系统等方面存在差异,导致室内外疏散路径无法无缝衔接;多数算法基于静态网络设计,缺乏对灾害演化过程的动态响应能力;疏散路径规划往往基于预设的危险区域或简化的风险假设,缺乏与精细化风险评估的有机结合。

\subsection{研究现状总结与问题归纳}

通过对国内外研究现状的系统梳理,可以发现当前研究在理论与方法层面均取得重要进展,但仍存在结构性问题:在灾前风险评估方面,主流路径包括基于历史灾情的经验规则方法与基于指标体系的定量分析方法,但普遍依赖二维逻辑与静态假设,缺乏对洪水等灾害物理演化的三维过程模拟;在灾中疏散方面,研究聚焦路径规划与人群行为模拟,但多数算法运行于简化二维平面或规则化网络,难以精准映射实际建筑拓扑与空间语义,尤其在室内外联通建模上易形成“语义孤岛”;在灾后损伤分析方面,虽有微尺度的构件级评估框架\cite{Amirebrahimi2016},但往往独立于城市级风险模型与疏散模型,难以形成贯通的数据链和业务闭环。在系统集成层面,已有工作尝试通过BIM+GIS融合、CIM可视化平台与数字孪生框架支持全周期任务管理,但多数仍处于原型或演示阶段,缺乏基于统一语义模型的可执行工作流与多环节协同验证。

在微尺度洪灾损伤评估(FDA)方向,传统方法大体可分为四类:事后调查法依赖灾后实地踏勘或遥感数据,成本高且对未来事件缺乏预测能力\cite{Downton2005};平均法(如Rapid Appraisal Method)以均值代替个体差异,难以刻画不同建筑的损伤变异性\cite{NRE2000};损伤曲线法凭经验曲线将淹没深度映射为损失,但对流速、持续时间等关键因素敏感且受地域制约\cite{Smith1994,Merz2010};详尽脆弱性评估法引入流体—结构相互作用建立力学模型,却常因数据缺失与几何简化导致结果不确定\cite{Kelman2002,Roos2003,Nadal2010}。这些方法侧重单栋或小范围分析,与城市级风险评估和疏散模型之间缺乏语义与数据的联动机制。

成因既包括数据标准不统一、空间语义与算法耦合不足,也涉及协同治理与实时信息链路不完善。现有研究在灾前评估阶段存在平面化、静态化倾向;在灾中疏散阶段存在语义脱节与空间割裂;在灾后损伤阶段缺乏与风险模型、疏散模型的联动;在系统集成层面存在流程断裂与实证不足。技术发展趋势表明,从单一技术向综合集成发展,从静态建模向动态仿真发展,从离线分析向实时应用发展,从技术导向向应用导向发展。主要研究空白包括跨尺度统一建模的理论与方法仍需完善;风险评估、疏散规划与损伤复盘的一体化框架缺乏;CIM在大规模实际场景中的应用验证不足;动态环境下的实时响应与恢复能力有待提升。


\section{研究目标和内容}

本研究旨在构建一个面向滨海城市洪涝场景的CIM驱动灾害风险评估、智能疏散与灾后损伤评估协同框架,形成“数据汇聚—风险计算—路径诱导—损伤复盘—协同指挥”的闭环流程,并以威海滨海应急服务中心为验证场景。围绕这一总体目标,研究重点回答三个相互递进的问题:(1) 如何实现多源数据的语义一致性表达,整合BIM、GIS、遥感与物联网数据并准确刻画风暴潮洪涝的三维动力学演化;(2) 如何构建风险驱动的室内外一体化疏散模型,将动态洪水风险映射到疏散网络,实现语义一致、拓扑连通且可实时更新的路径规划;(3) 如何在统一的CIM平台上实现风险评估、路径诱导与构件级损伤评估的联动,并通过典型滨海案例验证系统的实用性与可靠性。

针对上述问题,本研究设计了四个层层递进的研究内容:一是建立多源数据底座与高精度洪涝风险评估方法,构建三维水动力模拟与脆弱性指标体系,输出时空连续的危险度产品;二是提出MGNM增强策略与风险驱动代价函数,构建室内外联通的疏散网络并实现动态重规划;三是面向灾后恢复建立构件级洪灾损伤评估与三维可视化方法,实现与风险、疏散模型的多尺度联动;四是构建威海滨海应急服务中心综合验证流程,围绕洪涝模拟、疏散规划与构件损伤联动评估体系适用性。

\subsection{研究路线与数据支撑}

本研究遵循“理论分析—方法设计—系统开发—实验验证”的路线展开。数据层面,采集无人机倾斜摄影、激光雷达扫描、楼宇BIM、城市管网GIS、潮位与雨量监测等资料;对象层面,以滨海应急服务中心主体建筑及周边关键设施为研究单元;验证层面,通过历史风暴潮过程与应急演练脚本开展对比验证。每一阶段的核心成果在下一阶段得到继承,实现从机理模型到业务系统的逐步落地。

本研究的主要创新点体现在理论创新、方法创新、技术创新、应用创新四个方面。在理论创新方面,提出基于CIM的城市灾害全过程管理理论框架,建立多源数据融合的统一语义模型,发展覆盖城市、建筑与构件多尺度的空间建模理论;在方法创新方面,建立高精度三维水动力风险评估方法,设计语义驱动的室内外网络自动构建与风险感知疏散算法,并提出结合水动力作用与材料耐水性的构件级洪灾损伤评估方法;在技术创新方面,构建支持实时响应与灾后复盘的集成系统架构,开发面向多用户的可视化交互平台,实现风险评估、路径诱导与损伤评估的一体化应用;在应用创新方面,建立完整的案例验证框架,形成可复制推广的技术方案,推动城市韧性治理能力的系统提升。

本研究对"城市韧性治理"理论的贡献体现在理论层面、方法层面、实践层面三个方面。在理论层面,本研究首次系统性地将CIM技术范式与城市韧性治理理论相结合,提出"数字韧性治理"概念框架;基于复杂系统理论与CIM的多尺度建模能力,发展了多尺度韧性协同治理理论;将动态韧性边界控制理论引入城市治理领域,提出动态韧性边界治理理论。在方法层面,集成数字航空摄影测量、BIM–GIS融合与三维RANS水动力建模,构建具备构件级精度的洪涝风险评估体系;基于MGNM的自动构建与语义增强路径规划算法,实现从IFC到疏散网络的无缝转换;基于CIM平台的统一语义框架,构建了支持多部门、多层级、多角色协同参与的韧性治理方法。在实践层面,通过威海沿海城市典型案例开展全流程验证,构建了从数据采集、模型构建、风险评估到应急响应的完整技术体系;通过将建筑信息学、地理信息科学、水动力学、运筹优化等多学科知识进行系统集成,为跨学科融合研究提供了成功示范;为智慧城市建设提供了韧性导向的发展路径。

综合上述贡献,本研究在理论创新、方法突破与实践示范等方面为城市韧性治理学科发展做出了系统性贡献,对推动韧性治理理论的数字化转型与实践化应用具有重要的学术价值与现实意义。
