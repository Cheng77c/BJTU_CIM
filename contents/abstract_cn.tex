% ********************************************************************
% ****************** Free to modify the content below ******************
% ********************************************************************

随着城市化进程加速和极端天气事件频发,城市面临各种自然气象导致的灾害风险评估与人员疏散双重挑战。传统风险评估方法多基于二维平面模型且缺乏与疏散规划的有效耦合,难以满足现代城市精细化应急管理需求。本研究基于城市信息模型(City Information Modeling, CIM)构建多尺度灾害风险评估与智能疏散一体化框架,旨在解决城市应急管理中风险评估精度不足、室内外疏散路径割裂以及时空模型间协同性差等关键科学问题。

本研究首先融合无人机倾斜摄影、激光雷达等多源数据建立厘米级城市地形语义模型,基于三维雷诺平均纳维-斯托克斯(RANS)方程开发嵌套水动力求解器,采用6580万结构化网格单元实现高精度洪涝模拟,为城市尺度风险评估提供可靠的物理基础。其次,创新性地提出多用途几何网络模型(MGNM),将室内IFC语义与室外道路网络在CIM框架下自动拼接,构建统一的疏散网络表达,设计风险感知代价函数耦合动态洪水参数与人群密度,开发粗细结合的并行路径求解算法,将五栋互联建筑的疏散计算时间从485.2秒优化至127.6秒,计算效率提升74\%,实现了室内外一体化智能疏散路径规划。再次,基于Assembly-Based Vulnerability理论扩展BIM构件语义,建立涵盖静水压力、动水压力、浮力等多重作用机制的定量判据体系,实现从城市到构件的跨尺度损伤评估与三维可视化表达。以威海滨海应急服务中心为验证案例的研究结果表明,系统识别出整体构件损伤率27\%(门窗损伤率62\%,墙面损伤率54\%),估算直接经济损失51.4万元,功能恢复时间21天,集成系统可在14.6分钟内完成"风险评估—疏散规划—损伤分析"全链条演算,满足应急指挥的实时性要求。

本研究在理论上丰富了基于CIM的城市韧性理论体系,提出了多尺度风险评估与智能疏散的协同建模方法,在实践上为沿海城市应急管理部门提供了集风险预警、疏散引导、损失评估于一体的数字化工具,具有重要的工程应用价值和推广前景。

