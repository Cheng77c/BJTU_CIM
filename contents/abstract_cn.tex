% ********************************************************************
% ****************** Free to modify the content below ******************
% ********************************************************************

随着全球气候变化和快速城市化叠加,极端降雨、风暴潮等事件显著增多,城市洪涝对人口安全和关键基础设施造成的复合风险日益突出。传统以二维平面为基础的风险评估和以建筑单体为对象的疏散设计,难以刻画高密度建成环境中的三维水动力过程与跨建筑人群行为,导致“评估—疏散—损伤”三环节相互割裂,应急决策缺乏统一的数字底座与联动机制。为此,本文以城市信息建模(City Information Modeling, CIM)为核心,构建多尺度灾害风险评估与智能疏散一体化技术体系,实现灾前预警、灾中引导与灾后复盘的闭环支撑。

在理论层面,本文将复杂系统视角下的城市韧性治理与数字孪生城市理念引入 CIM 研究,提出“数字韧性治理”总体框架,形成覆盖物质、空间、性能、文化与时间的“五维—四视角(5D–4V)”建模方法,从场景化、参数化、互动化和智能化四个维度刻画城市对象的多层语义及其演化规律。通过构建统一的城市语义矩阵与对象关系模型,实现了从建筑构件—建筑单体—街区—城市的跨尺度表达,为后续水动力仿真、疏散网络与构件损伤评估提供一致的数据与知识基础。

在风险评估环节,本文融合无人机倾斜摄影、机载激光雷达、CityGML 与 BIM/IFC 等多源数据,建立厘米级精度的城市三维地形与建筑语义模型,形成支撑 0.5 m 分辨率风险栅格计算的数据底座。基于三维雷诺平均纳维–斯托克斯(RANS)方程开发嵌套水动力求解器,采用 6580 万结构化网格刻画建筑群周边复杂流场,相较二维浅水模型显著提升了淹没范围与压力估计精度,误差分别降低 10.8\% 和 16.0\%。在此基础上,构建包含物理暴露、基础设施脆弱性与社会经济敏感性的综合指标体系,生成时空连续的洪涝危险度与人口暴露图,为疏散策略与损伤评估提供物理约束。

在智能疏散环节,本文提出多用途几何网络模型(Multi-purpose Geometric Network Model, MGNM),从 IFC 自动抽取房间、走廊、门窗、楼梯等语义对象及其拓扑关系,构建与建筑几何高度一致的室内通行网络,并通过出入口与街道节点与室外道路网自动拼接,形成室内外一体化可达图。进一步设计耦合水深、流速和人群密度的风险感知代价函数,提出粗细结合的并行路径求解策略,实现城市级网络的快速更新与局部重规划。以五栋互联建筑为例,路径计算时间由 485.2 s 降至 127.6 s,计算效率提升约 74\%;在典型洪涝情景下,风险驱动疏散使总疏散时间缩短 15.3\%,处于高风险边上的路径占比降至 9.4\%,显著增强了人员撤离的安全性与鲁棒性。

在灾后损伤评估环节,本文在 Assembly-Based Vulnerability 理论基础上扩展 BIM 构件语义,建立静水压力、动水压力和浮力等作用与构件几何、材性之间的定量关系,提出面向单构件的损伤判别准则及维修/更换成本映射方法。综合构件损伤率、功能恢复时间和投资回收比等指标,构建构件级洪灾损伤评估与三维可视化框架,可按构件类型、损伤等级和经济损失进行过滤、统计与空间查询,实现“从风险场到构件”的精细化量化分析。以威海滨海应急服务中心为例,系统识别出整体构件损伤率约 27\%(其中门窗 62\%、墙面饰材 54\%),估算直接经济损失 51.4 万元,关键功能预计 21 天恢复,为制定分阶段韧性改造计划和资产管理清单提供量化依据。

在系统集成与协同应用方面,本文将高精度水动力模拟、MGNM 疏散网络和构件级损伤模块嵌入统一 CIM 平台,构建“数据采集—语义治理—风险评估—路径规划—损伤分析—协同指挥”的闭环协同系统。系统采用分层微服务架构和事件驱动机制,支持风险栅格、疏散网络与损伤结果的在线更新和联动调用,能够在 14.6 min 内完成威海滨海应急服务中心“风险评估—疏散规划—损伤分析”全流程演算,并通过三维可视化界面向指挥员和公众终端同步展现风险态势与引导路径。实际部署与演练表明,该系统在分钟级时效、跨部门协同和场景直观性方面均满足应急管理的业务需求。

综上,本文在理论上系统提出基于 CIM 的城市灾害全过程管理与数字韧性治理框架,在方法上形成“高精度三维水动力风险评估—风险驱动室内外一体化疏散—构件级洪灾损伤评估”的多尺度联动技术链,在工程上实现城市应急管理的原型系统并完成真实场景验证。研究成果为构建集风险预警、智能疏散与灾后恢复于一体的数字化防灾体系提供了可复制的技术路径,也为未来在多灾种联动、实时数据同化和行为机理精细建模等方面的深入研究奠定了基础。



