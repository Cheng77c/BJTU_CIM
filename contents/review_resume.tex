\begin{ResumePublications}
	\begin{ResumeList}
		\item 作者简历\par
		李栋,男,1979年8月出生于河南扶沟,1997—2001年就读于兰州铁道学院信息与电气工程学院通信工程专业,获得工学学士学位;2004—2008年就读于清华大学电子系电子与通信工程专业,获得工程硕士学位;2021年就读北京交通大学土木建筑工程学院攻读工程博士学位。2001—2021年就职于中铁通信信号勘测设计院有限公司;2021至今就职于中铁第六勘察设计院集团有限公司(中国中铁智慧城市研发中心)。
		\item 发表论文
		\begin{PublicationsList}
			\item (第一作者,EI)基于机器视觉的高鲁棒轨道表面缺陷检测方法\hspace{0em}[J].铁道工程学报,2024,41(05):11-18.
			\item (第一作者,SCI) Enhanced Structure-from-Motion 3D Reconstruction through Deep Learning Feature Fusion and Optimization\hspace{0em}[J]. Tehnički vjesnik / Technical Gazette(技术公报)TV-20240908001975,第32卷/第6期.
			\item (第二作者,EI会议论文) Load Balancing Oriented Quick Partitioning Algorithm of Simulation Road Networkfor Parallel Traffic Systems\hspace{0em}[C]. Proceeding of 2022 International Conference on Wireless Communications, Networking and Applications (WCNA 2022). WCNA 2022. Lecture Notes in Electrical Engineering, vol 1059. Springer, Singapore.doi:10.1007/978-981-99-3951-0—48.
			\item (第三作者,EI会议论文) Deep reinforcement learning method for POMDP based tram signal priority\hspace{0em}[C]. 2023 IEEE 26th International Conference on Intelligent Transportation Systems (ITSC), Bilbao, Spain, 2023, pp. 229-234, doi: 10.1109/ITSC57777.2023.10422037.
			\item (第四作者,SCI) Attention-guided cascaded network for predicting tunnel surrounding rock properties using measurement-while-drilling data\hspace{0em}[J]. Automation in Construction,Volume 177,2025.
		\end{PublicationsList}
		\item 参与科研项目
		\begin{PublicationsList}
			\item 开放式CIM深度学习数字基座架构与服务平台原型研究(中国中铁科技研究开发计划:2023-重大-21),主持.
			\item 数字孪生城市云场景微代码构建平台共性技术研究(中铁六院科技研究开发计划:KY-2023-03),主持.
			\item 生成式轨道交通运维场景关键技术研究(天津市科技领军企业重大创新项目:24YDLQYS00110),主持.
			\item 基于数字孪生的轨道交通智慧运维关键技术研究(国家铁路局2024年重大课题:KF2024-077),主持.
			\item 轨道交通生成式人工智能智慧运维关键技术研究(中施企协2024年重点研发项目:2024-A-024),主持.
			\item 轨道交通一体化安全态势评估与预警技术(中铁六院科技研究开发计划:KY-2024-07),主持.
			\item 面向虚拟编组动态防护控制的列车时空轨迹安全预测方法研究(国家自然科学基金项目:2023:52372309),参加.
			\item 基于生成式AI的工程结构设计软件(中国中铁 2024-2025 信息化建设任务),参加.
			\item 基于有限元的工程设计仿真一体化软件(中国中铁 2024-2025 信息化建设任务),参加.
			\item 基于勘察设计大模型的生成式设计关键技术初步研究(中铁六院科技研究开发计划:KY-2024-01),参加.
			\item 隧道超前地质信息智能解译与工程智能辅助决策技术研究(中铁六院科技研究开发计划:KY-2024-08),参加.
		\end{PublicationsList}
		\item 专利
		\begin{PublicationsList}
			\item (第一发明人,发明专利)城市轨道交通运力与客流适应性评估方法、系统及设备,专利号:ZL 2023 11490207.7.
			\item (第一发明人,发明专利)一种轨道缺陷检测系统鲁棒性提升方法,专利号:ZL 2024 10315833.0.
			\item (第一发明人,发明公开)一种安全敏感的五维四视角城市信息模型构建方法,申请号:202510474601,公开号:CN120338625A.
			\item (第一发明人,发明公开)基于区块链与MPC的联邦学习城市信息模型及其构建方法,申请号:CN202510477049.4,公开号:CN120277721A.
			\item (第三发明人,发明专利)基于Event-B的Java代码自动生成中内存安全验证方法,专利号:ZL 2025 1 0272147.4.
			\item (第四发明人,发明公开)基于区块链与差分隐私的联邦学习城市信息模型系统信息安全保护方法,申请号:CN202510484264.7,公开号:CN120337292A.
		\end{PublicationsList}
		\item 标准规范
		\begin{PublicationsList}
			\item (国家标准,主编)GB/T 45844-2025 智慧城市基础设施 开发和运营通用框架.
			\item (国际标准,联合主导,已发布)IEC63302-1  Intelligent operations center for Smart Cities–Part 1 : High Level Analysis.
			\item (国际标准,联合主导,已发布)ISO 37176  Smart community infrastructure—Responsiveness assessment and maturity model.
			\item (国际标准,主导,编制中)IEC63302-2  Intelligent operations center for Smart Cities–Part 2 : Use Case Analysis.
			\item (国家标准,参编)GB/T 44493-2024 智慧城市基础设施 智慧交通中城市停车位匹配实施指南.
			\item (国家标准,参编)GB/T42882-2023 智慧城市运行模型 应对城市突发公共卫生事件的应用指南.
		\end{PublicationsList}
	\end{ResumeList}
\end{ResumePublications}

